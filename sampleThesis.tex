\documentclass[phd,tocprelim]{cornell}
%% Added to avoid the \ifpdf name clash error
\let\ifpdf\relax
%
% tocprelim option must be included to put the roman numeral pages in the
% table of contents
%
% The cornellheadings option will make headings completely consistent with
% guidelines.
%
% This sample document was originally provided by Blake Jacquot, and
% fixed up by Andrew Myers.
%
%Some possible packages to include
\usepackage{graphicx,pstricks}
\usepackage{graphics}
\usepackage{moreverb}
\usepackage{subfigure}
\usepackage{epsfig}
\usepackage{subfigure}
\usepackage{hangcaption}
\usepackage{txfonts}
\usepackage{palatino}

%if you're having problems with overfull boxes, you may need to increase
%the tolerance to 9999
\tolerance=9999

\bibliographystyle{plain}
%\bibliographystyle{IEEEbib}

\renewcommand{\caption}[1]{\singlespacing\hangcaption{#1}\normalspacing}
\renewcommand{\topfraction}{0.85}
\renewcommand{\textfraction}{0.1}
\renewcommand{\floatpagefraction}{0.75}

%% \title {Structural variations in cancer genomes: representation, inference, and applications}
\title{Illuminating rearranged cancer genome structures through genome graphs}
\author {Xiaotong Yao}
\conferraldate {Aug}{2021}
\degreefield {Ph.D. Computational Biology}
\copyrightholder{Xiaotong Yao}
\copyrightyear{2021}

\begin{document}

\maketitle
\makecopyright

\begin{abstract}
Cancer genomes harbor structural variations (SV), producing various drivers and reflecting . Whole genome sequencing (WGS) characterizes SVs in the form of copy number aberrations (CNA) and junctions. Many simple and complex patterns of SVs have been discovered. Yet, one of the biggest hurdles to analyze SVs is the lack of flexible and general framework to account for the complexity of SVs. SVs inherently alter the coordinate system of the reference genome, thus the interpretation of any junction is dependent on other overlapping junctions. Plus, with short-read WGS, we can only rely on relatively local sequence readouts, making it challenging to reconstruct the long range linear structures of the DNA. To these ends, I present genome graph as a general framework to represent rearranged genomes, which treats genomic sequences as directed graphs, where vertices are single-stranded DNA segments and edges are 3'-5' phosphodiester bonds joining adjacent vertices. Within this framework, I then present a mixed-integer programming algorithm Junction Balance Analysis (JaBbA) to infer integer copy numbers (CN) for vertices and edges from the WGS of a tumor sample. I show that JaBbA not only achieves more accurate CN estimation, more complete genome graphs than other methods, its robust recapitulation of junction copy numbers (JCN) serves a pivotal role in discovering distinct patterns of complex rearrangements from large-scale pan-cancer WGS studies. Finally, to investigate the SV outcomes of natural telomere crisis, an inevitable obstacle most cancer must overcome and has been linked to complex SV patterns, I use the genome graph framework to reconstruct the SV events in lineages of human fibroblast cells surviving natural telomere crisis by induced telomerase expression. To sum up, genome graph with the Junction Balance Analysis algorithm enable a general, robust analysis framework that help elucidate the complexity of SVs in cancer genome.
\end{abstract}

\begin{biosketch}
Xiaotong Yao has had strong interest in biology from an early age and later developed an appreciation for using computational methods to answer biological questions. His interest for computational biology started with systems biology and synthetic biology by participating in the 2011 International Genetically Engineered Machine Competition. Subsequently, he curated lipid metabolic pathways in the \textit{Streptomyces avermitilis} metabolic network. To get formally trained in computational biology, he joined the Master's program in Bioinformatics and Systems Biology at the Biology Department of New York University, where he built predictive models for protein sumoylation from public protein function databases. Following a passion for precision medicine, in 2015, he became a PhD student in the Tri-institute Program in Computational Biology and Medicine at Weill Cornell Medicine, focusing on cancer genomics, and in 2016 he joined Dr. Marcin Imielinski's lab to persue his dissertation research in using genome graphs to model complex structural variations in cancer whole genomes.
\end{biosketch}

\begin{dedication}
To Rosalind Yao and Shan Huang.
\end{dedication}

\begin{acknowledgements}
Marcin
Kevin
Julie
Aditya
Committee
Collaborators
Program
Sample donors
\end{acknowledgements}

\contentspage
\tablelistpage
\figurelistpage
\abbrlist

SV -- structural variation

WGS -- whole genome sequencing

CN -- copy number

CNA -- copy number aberration

JaBbA -- Junction Balance Analysis

\symlist

\normalspacing \setcounter{page}{1} \pagenumbering{arabic}
\pagestyle{cornell} \addtolength{\parskip}{0.5\baselineskip}

%%%%%%%%%%%%%%%%%%%%%%%%%%%%%%%%%%%%%%%%%%%%%%%
%% Chapter 1: introduction
%%%%%%%%%%%%%%%%%%%%%%%%%%%%%%%%%%%%%%%%%%%%%%%
%% To motivate what comes next.
\chapter{Introduction}

\section{Complexity of SVs in the cancer genomes}
Genomic instability is a hallmark of cancer. With whole genome sequencing (WGS), we are profiling large numbers of cancer genomes revealing ubiquitous yet heterogeneous patterns of single nucleotide variations (SNV), insertions and deletions (INDEL), and structural variations (SV). However, the discoveries in SNVs and INDELs have been outpacing that of SVs, exemplified by the distillation of mutation signatures from large-scale mutation profiles described in local sequence context of the substitution. Even though we have known for a long time that cancer genomes present widespread aneuploidy and rearrangements, it still remains challenging to characterize the full spectrum of SVs in cancer.

There are many reasons to this, but a fundamental one is that the SVs in cancer genomes are complex and there lacks a suitable framework to account for it. Cancer WGS studies have revealed a plethora of complex SV patterns that can involve up to hundreds of junctions. A prominent example is chromothripsis, theorized to originate from  shattering a chromosome arm and erroroneously repaired in random order with interspersed loss of genetic materials. Such processes can generate up to hundreds of junctions with random location and orientations. Setting aside the debate about the definition of chromothripsis for later chapters, chromothripsis-like events are estimated to be seen in as many as xx\% of pan-cancer genomes (cite PCAWG SV and Park).

In sharp contrast, most studies on genomic structural variations have been geared towards germline genomes, which are dominated by simple SV classes, exemplified by tandem duplications, simple deletions, and inversions. Although more complex patterns have also been detected and characterized, they are (expectedly) rare relative to the simpler ones. Since each simple SV only contains one (tandem duplications and deletions) or two junctions (inversions and balanced translocations), it is usually convenient to apply the substitution model that has long been the de facto standard in representing and storing small genomic variants, such that one variant is represented as an edit of a fixed location of the reference genome and in most cases each variant can be interepreted independently from the other variants.

The limitation of this approach is visible even in the simplest form of complexity. Considering the toy model in (Fig 1A, adapted from MI/JM's review), where a unrearranged genomic contig \textit{ABCDE} underwent two rounds of tandem duplications, yet when mapped to the reference genome, the second junction appeared to be consistent with a deletion. As a result, the development of a general framework to analyze complex SVs is needed.

\section{Representation of structural variations with genome graphs}
Graphs have been widely used to model genomic sequences for decades, with \textit{de novo} assembly as one of the fields most heavily reliant (citation SGA, Fermi, etc.). One way to describe sequence as graphs is that each vertex is the sequence of a contiguous DNA segment (contig) and following the adjacencies among vertices one can thread a path (or cycle) to obtain longer sequence and eventually approach a genome. In resequencing studies, we map reads to locations in an existing reference genome, and identify variants based on the discrepancy between reads and the mapped reference sequence. Nevertheless, we can apply a similar idea of modeling, now replacing actual DNA sequences to the genomic intervals they map to in the reference. There have been several studies employed the concept of \textit{interval graph} to model the somatic SVs in cancer genomes. which treats xxx (describe the interval graph models).

\section{Reconstruction of genome graphs}
While conceptually interval graph has been an advancement over the over-simplistic substitution model, the cancer genomics research community has not fully adopted it as the mainstream framework for SV analysis. The main reason is that the numeric quality of the copy numbers in the reconstructed graphs have not shown high enough fidelity for the heterogeneous lanscape of complex somatic SVs, as well as varying sample purity. There have been several approaches proposed with slightly different 

\section{From patterns to mechanisms to etiology}
The ultimate goal of studying patterns of SVs is to find common classes out of large panels of tumors, and link them back to the possible mechanisms from which specific SVs result, for example, telomere crisis.

Early efforts in this realm successfully identified many distinct patterns.

\section{Evolution of SVs after telomere crisis}

%%%%%%%%%%%%%%%%%%%%%%%%%%%%%%%%%%%%%%%%%%%%%%%
%% Chapter 2: formalize genome graphs and JaBbA
%%%%%%%%%%%%%%%%%%%%%%%%%%%%%%%%%%%%%%%%%%%%%%%
\chapter{Junction-balanced genome graphs represent structurally altered genomes}
In this chapter, I give formal definitions of a genome graph, a junction balanced genome graph, and Junction Balance Analysis to infer integer copy numbers from short-read whole genome sequencing data. I also establish the correspondance between a walk on a genome graph to the underlying karyotype and show genome graphs can encapsulate a diverse array of putative functional events including fusion genes and enhancer hijiacking. Last but not least, taking advantage of our novel interactive genome browser \textit{gGnome.js}, I show that our complete toolset can serve as an intuitive portal for oncologists to fully take advantage of cancer WGS data.

\section{Genome graph as a general data structure to represent rearranged genome}
% definition in as plain language as possible
% need to define: reference genome, interval, signed interval, junction
Based on a reference genome, we define genome graph $G = (V, E)$, where the vertices $V$ is a set of signed intervals ($+$ if referring to the sequence of the reference, $-$ if reverse complement, more details in Appendix A), and edges $E$ is a set of directed adjacencies representing the 3-5 phosphodiester bonds between vertices. Since genomic DNA is a pair of reverse complement strands, we restrict $G$ to be skew-symmetric (see formal definitions in Methods) with the mapping function $r$ "reverse complement" \cite{Goldberg1996-qm}, such that for any vertex $v \in V$, there exists its \textit{symmetric} vertex $\bar{v} = r(v) \in V$, and for any edge $e = (v_1, v_2) \in E$, $\bar{e} = r(e) = (\bar{v_2}, \bar{v_1}) \in E$. Intuitively, each symmetric pair of vertices represent a double stranded DNA segment within the reference genome. Subsequently, there are two types of edges. \textit{REF} edges are the ones that are connecting vertices adjacent in the reference genome, and \textit{ALT} are neo-adjacencies that are not present in the reference genome resulting from rearrangements (Fig2). A reverse complement pair of vertices $\{v, \bar{v}\}$ compose a double-stranded DNA segment (segment for short); a reverse complement pair of edges $\{e, \bar{e}\}$ compose a junction. Analogous to an assembly graph, to produce a substring in the genome, is equivalent to traveling through a walk $w = (v_1, e_1, v_2, e_2, ..., e_k, v_k)$, where each edge $e_i = (v_i, v_{i+1}), i \in 1, 2, ..., k-1$, at the same time its reverse complement walk $\bar{w} = (\bar{v_k}, \bar{e_k}, \bar{v_{k-1}}, ..., \bar{e_1}, \bar{v_1})$.

% 
Following the above definitions, except for whole genome/chromosome/contig gain or losses or foreign sequence insertions, every structural change of a genome can be effectively seen as generating new adjacencies (ALT edges) between previously non-adjacent vertices. In other words, assuming we know all the ALT edges resulted from the 

% Note that this abstraction does not make assumptions about the mechanism by which the new adjacencies arise, rather it describes the difference of states after an SV event.

Extending the properties of $r$ to any function of vertices or edges, we call a function $f(v), v \in V(G)$ or $f(e), e \in E(G)$ as skew-symmetric if $f(v) = f(\bar{v})$, $f(e)=f(\bar{e})$. One such function is copy number $\kappa : G \rightarrow \mathbb{N}$, as for double stranded DNA the copy number of one strand should always be equal to its reverse complement.

Intuitively, a walk along on a genome graph maps to a DNA sequence and a genome graph is a compilation of all possible linear sequences that could result from rearranging the reference genome assuming the junctions are completely known. This idea is similar to the assembly graph in \textit{de novo} assembly, except we are pre-defined within a linear reference genome and adding further rearrangements.

\section{Implementation of genome graphs in gGnome package}
% noteworthy implementation details
We implemented 

% construction
In the most basic form, a genome graph can be initiated with a reference genome alone, where each node is a chromosome and no edges exist (\texttt{gG()}). On top of that, we can segment the whole genome based on a set of breakpoint coordinates, by automatically connecting two consecutive nodes with a (pair of reverse complement) REF edge (\texttt{gG(breaks = breakpoints)}). When a ALT junction emerges, it joins two breakends that are not adjacent in reference genome to form new (pair of reverse complement) ALT edges (\texttt{gG(juncs = junctions)}).

More often in practical scenarios, we start from a copy number-annotated genome graph inferred from WGS. gGnome allows one to parse the results from most of the existing genome graph reconstruction methods (JaBbA, ReMixT, Weaver, PREGO, RCK, CouGaR, AmpliconArchitect). 

% subgraph
With a genome graph, we can make a series of queries based on vertex and edge metadata.

\section{Neo-connectivities on a genome graph encode putative functional events}

As explained in the previous sections, a walk (and its reverse complement walk) on a genome graph maps to a linear DNA sequence. If there exists a possible walk between two vertices it means the genomic elements within the two vertices may be connected on a contiguous DNA segment in the sample, if and only if all the ALT edges along the walk are phased in cis. A pair of ALT edges phased in cis means there are at least one copy of them found on the same DNA molecule; while none such copy is found they are deemed phased in trans. In reality, due to the large size of the genome and the limited read-length or effective range (e.g. for linked reads), complete phasing of all junctions is near impossible for most cases. However, the connectivities between genomic regions on the un-phased or partially phased genome graphs can serve as an approximation. 

As we established in earlier sections, a complete genome graph should enclose all the walks corresponding to the studied genome, so if any copy of two genomic loci are on the same part of a DNA molecule, that segment should have a corresponding walk on the complete genome graph. Based on this observation, the new genomic distance after rearrangements between any copy of two loci is summing over the width of vertices plus the within-vertex distances to the incident ends along the walk for the starting and terminating vertices. Thus, even when it is not possible to exactly decompress the graph to the walks consistent with the actual karyotypes, we know that the real shortest genomic distance between any copy of two loci should not be shorter than the shortest path distance on the complete genome graph.

Admittedly, there may never be a complete genome graph constructed with the current technologies, and it is always possible that we failed to identify certain junction that cut some genomic distances even shorter. Nevertheless, as shown in (Behr 2021, BioRxiv), the state of the art short-read WGS should be effective at catching the majority of somatic rearrangements. Plus, even with the imperfect genome graphs, we can already expand the repertoire of functional genomic events. We use this approximation to show that potential functional events of arbitrary complexity can be represented as walks on genome graphs, exemplified by bridged fusion genes and enhancer hijacking. To do that, I reconstructed JBGG from PCAWG consensus somatic junctions and copy numbers with the function \textit{balance} in the gGnome package.

Some fusion genes have long been established as cancer drivers (citation) and even successful drug targets (citation); some have also been shown to encode chimeric proteins that can trigger immune responses as neoantigens (citation). Expectedly, most fusion genes found to date are created by one somatic junction, as it requires fewer junctions to create and has simpler topology to detect. However, in theory, there should be no biological restraint against a more complex, multi-junction fusion gene. Indeed, recent large pan-cancer whole genome and transcriptome profiling efforts have identified numerous such examples termed bridged fusions (citation PCAWG). Naturally following our above definition, any fusion genes, no matter how many somatic junctions is involved, can be represented as a walk $w = (v_1, e_1, v_2, ..., e_{k-1}, v_k)$ on the corresponding genome graph, where the 5’ partner is at the outgoing end of $v_1$ and the 3’ partner is at the receiving end of $v_k$ (Fig x). Here I show an example walk of xxx—xxx fusion found by the PCAWG Transcriptome working group (citation) along with the supporting reads from the RNA-seq. By our graph-based SV event classification, this junction was part of a large scale xx (described in the next chapter).

Another emerging class of functional SV events is enhancer hijackings, where a distal active enhancer is juxtaposed to a proto-oncogene and facilitates its transcription. On a genome graph, they can be detected as shortened genomic distances through paths over somatic junctions. One such event that has been characterized and functionally validated in thyroid carcinoma cell lines is THADA-IGF2BP3, where the enhancer near the constitutively thyroid-expressed THADA gene was juxtaposed to and subsequently activates IGF2BP3 expression (citation and read the paper deeper). This event has also been found to be recurrent in the genomes of thyroid carcinoma in PCAWG independently. I demonstrate the walks corresponding to these events (Fig xx), one (patient) of which arise from a more complex chromoplexy(?) event. Using this strategy, I found more instances of thyroid carcinoma samples with over-expressed (quantile>0.95) IGF2BP3 genes without amplification yet rearranged near 

\section{Inferring copy numbers on genome graphs from whole genome sequencing with Junction Balance Analysis}


\section{Formulation of the mixed-integer quadratic programming problem}


\section{Data preprocessing and graph partitioning}

% \section{Automatic tuning of convergence criteria to improve the performance of harder problems}

\section{Post processing and allelic CN fitting}

\section{JaBbA robustly produce accurate copy numbers for DNA segments and junctions}

\section{Interactive visualization of genome graphs in arbitrary genomic windows}

\section{Discussion}

%%%%%%%%%%%%%%%%%%%%%%%%%%%%%%%%%%%%%%%%%%%%%%%
%% Chapter 3: discovery of novel SV patterns
%%%%%%%%%%%%%%%%%%%%%%%%%%%%%%%%%%%%%%%%%%%%%%%
\chapter{Distinct Classes of Complex Structural Variation Uncovered across Thousands of Cancer Genome Graphs}
In this chapter I will describe the discovery of three distinct types of complex SV events 

\section{Constructing pan-cancer genome graphs}

\section{Low JCN clusters of deletion-like junctions form rigma}

\section{Rigma is preferentially affecting late replicating, fragile sites in gastrointestinal tumors}

\section{Low JCN clusters of tandem duplication-like junctions form pyrgo}

\section{Pyrgo is overrepresented in early replicating regions and superenhancers}

\section{Amplified subgraphs with high JCN junctions show three stable subtypes}

\section{Tyfonas is massive amplicon heavily burdened by junctions of heterogeneous JCNs}

\section{Clusters of pan-cancer patients based on SV event burdens show differential prognosis and is associated with tumor types and genetic backgrounds}

\section{Discussion}


%%%%%%%%%%%%%%%%%%%%%%%%%%%%%%%%%%%%%%%%%%%%%%%
%% Chapter 4: SV evolution after telomere crisis
%%%%%%%%%%%%%%%%%%%%%%%%%%%%%%%%%%%%%%%%%%%%%%%
\chapter{Structural variant evolution after telomere crisis}
In this chapter I set out to capture the SV events in human cell lines directly resulting from natural telomere crisis

\section{Both simple and complex SVs detected in previously described post-crisis cell lines}

\section{An \textit{in vitro} model of natural telomere crisis in human lung fibroblast cell line MRC5}

\section{Screening for structural alterations in post-crisis MRC5 clones using low-pass WGS}

\section{Integrate SVs and SNVs to Reconstruct phylogeny of post-crisis clones with high-pass WGS}

\section{One allele of chromosome 12p with the shortest telomere length is the origin of genome instability during telomere crisis}

\section{Discussion}

\appendix
\chapter{Appendix A}

\section{Definition of an interval on the reference genome}

\cite{aiw}

\begin{equation}
    k_1=\frac{\omega }{c({1/\varepsilon_m + 1/\varepsilon_i})^{1/2}}=k_2=\frac{\omega
    sin(\theta)\varepsilon_{air}^{1/2}}{c}
\end{equation}


\bibliography{sampleThesis}

\end{document}